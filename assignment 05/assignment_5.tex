\documentclass{article}
\usepackage[utf8]{inputenc}
\usepackage{titling}

% Page size and margins
\usepackage[a4paper,top=2cm,bottom=2cm,left=3cm,right=3cm,marginparwidth=1.75cm]{geometry}

% Language setting
\usepackage[english]{babel}

% Useful packages
\usepackage{amsmath}
\usepackage[colorlinks=true, allcolors=blue]{hyperref}
\usepackage{minted}

% Header
\usepackage{fancyhdr}
\pagestyle{fancy}
\fancyhead{} % empties all fields
\fancyhead[R]{Francesca Grimaldi}
\fancyhead[L]{IDATT2503 - Exercise 5}

% Images
\usepackage{graphicx}
\graphicspath{ {images/} }
\usepackage{float}

% Tables
\usepackage{tabu}
\usepackage{caption} 
\captionsetup[table]{skip=2pt}
\usepackage[table]{xcolor}
\usepackage{array}

% Lists
\usepackage{listings}
\usepackage{enumitem}
\setlist{topsep=2pt, itemsep=2pt, partopsep=2pt, parsep=2pt}

% renew/new command
\newcommand{\subtitle}[1]{%
  \posttitle{%
    \par\end{center}
    \begin{center}\large#1\end{center}
    \vskip0.5em}
}

% Title and info
\title{%
    \huge Assignment 5}

\subtitle{%
    IDATT2503 - Security in programming and cryptography \\
    Fall 2023
    }

\author{%
  Francesca Grimaldi
}

\date{}


\begin{document}

% TITLE
\maketitle


% INDEX
% \tableofcontents

% FUZZING
\section{Fuzzing}
Following the instructions of the \href{https://gitlab.com/ntnu-tdat3020/fuzzing-example}{\texttt{fuzzing-example}}, I executed the following steps:
\begin{itemize}
    \item[1.]{Installed \texttt{cmake} and \texttt{clang} from terminal}
    \item[2.]{Created a \texttt{build} directory inside that of my C project}
    \item[3.]{Moved into the new build directory}
    \item[4.]{Configured the build system for the project: set the C compiler to be Clang and ran the CMake build system}
    \item[5.]{Compiled the source code and created executable files based on the \texttt{CMakeLists.txt} ones}
    \item[6.]{Ran the fuzzer test with a time limit of 60 seconds}
    \item[7.]{Ran the utility tests with assertions}
\end{itemize}

\begin{figure}[H]
    \centering
    \includegraphics[width=0.85\textwidth]{img/initial-setup.png}
    \caption{Initial setup}
    \label{fig:initial-setup}
\end{figure}

The first fuzzing showed that there were memory leaks in the program.

\begin{figure}[H]
    \centering
    \includegraphics[width=0.85\textwidth]{img/fuzzing-memory-leak1.png}
    \caption{Fuzzing showing memory leaks (a)}
    \label{fig:fuzzing-memory-leak1}
\end{figure}

\begin{figure}[H]
    \centering
    \includegraphics[width=0.85\textwidth]{img/fuzzing-memory-leak2.png}
    \caption{Fuzzing showing memory leaks (b)}
    \label{fig:fuzzing-memory-leak2}
\end{figure}

After the bug fixes, the results are the following:

\begin{figure}[H]
    \centering
    \includegraphics[width=0.85\textwidth]{img/fuzzing-after-fixes.png}
    \caption{Fuzzing after fixes}
    \label{fig:fuzzing-after-fixes}
\end{figure}

\begin{figure}[H]
    \centering
    \includegraphics[width=0.7\textwidth]{img/testing.png}
    \caption{Running tests does not produce errors}
    \label{fig:testing}
\end{figure}

% CONTINUOUS INTEGRATION SOLUTION
\section{CI solution}
To set up a Continuous Integration solution for fuzzing with address sanitizer, I used the GitHub Actions platform.

For this, I created a repository with the project and a \texttt{.github/workflows} folder, in which I added the YAML file \texttt{ci-fuzzing.yml}. In this I defined the workflow, including the instructions necessary for the setup and those for the fuzzing and testing.

After committing the above-mentioned file to the repository, it gives the following output:

\begin{figure}[H]
    \centering
    \includegraphics[width=0.85\textwidth]{img/ci-fuzzing.png}
    \caption{Fuzzing in the CI solution}
    \label{fig:ci-fuzzing}
\end{figure}

\begin{figure}[H]
    \centering
    \includegraphics[width=0.85\textwidth]{img/ci-fuzzing2.png}
    \caption{Final results in GitHub Actions}
    \label{fig:ci-fuzzing2}
\end{figure}


\end{document}