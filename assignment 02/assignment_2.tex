\documentclass{article}
\usepackage[utf8]{inputenc}
\usepackage{titling}

% Page size and margins
\usepackage[a4paper,top=2cm,bottom=2cm,left=3cm,right=3cm,marginparwidth=1.75cm]{geometry}

% Language setting
\usepackage[english]{babel}

% Useful packages
\usepackage{amsmath}
\usepackage[colorlinks=true, allcolors=blue]{hyperref}
\usepackage{minted}

% Header
\usepackage{fancyhdr}
\pagestyle{fancy}
\fancyhead{} % empties all fields
\fancyhead[R]{Francesca Grimaldi}
\fancyhead[L]{IDATT2503 - Exercise 2}

% Images
\usepackage{graphicx}
\graphicspath{ {images/} }
\usepackage{float}

% Tables
\usepackage{tabu}
\usepackage{caption} 
\captionsetup[table]{skip=2pt}
\usepackage[table]{xcolor}
\usepackage{array}

% Lists
\usepackage{listings}
\usepackage{enumitem}
\setlist{topsep=2pt, itemsep=2pt, partopsep=2pt, parsep=2pt}

% renew/new command
\newcommand{\subtitle}[1]{%
  \posttitle{%
    \par\end{center}
    \begin{center}\large#1\end{center}
    \vskip0.5em}
}

% Title and info
\title{%
    \huge Assignment 2
    }

\subtitle{%
    IDATT2503 - Security in programming and cryptography\\
    Fall 2023
    }

\author{%
  Francesca Grimaldi
}

\date{}

\begin{document}

% TITLE
\maketitle


% INDEX
% \tableofcontents


% PYCALC CHALLENGE
\section{PyCalc challenge}

\subsection{Summary}
After connecting to \texttt{10.9.8.1:5006}, the calculator program welcomes the user suggesting to type \texttt{help} to display all the available commands. The possible ones are the following:
\begin{itemize}
    \item{The first, \texttt{echo}, prints out the argument inserted by the user to standard output. For instance, typing \texttt{echo hello} would print \texttt{hello} back to the user.}
    \item{The second one, \texttt{op}, evaluates a simple function. The result of typing \texttt{op 3+1} would therefore be seeing \texttt{4} printed out.}
\end{itemize}
\begin{figure}[H]
    \centering
    \includegraphics[width=0.7\textwidth]{pycalc/pycalc intro.png}
    \caption{PyCalc operations}
    \label{fig:PyCalc intro}
\end{figure}
The \texttt{op} command calculates the sum and other equations using a vulnerable Python function, called \texttt{eval()}. In general, it evaluates and executes any specified expression, if it is a legal Python statement. The problem is that accepting untrusted input to \texttt{eval()} can allow for arbitrary code execution by malicious users.\\
Writing
\begin{minted}{shell}
    op __import__('subprocess').getoutput('ls')
\end{minted}
reveals the list of all the directories and files present in the directory in which the calculator program is running.
\begin{figure}[H]
    \centering
    \includegraphics[width=0.7\textwidth]{pycalc/pycalc directories.png}
    \caption{List of files and directories}
    \label{fig:PyCalc directories}
\end{figure}
It is possible to notice that there is also a text file called \texttt{flag.txt}. Analogously, in order to show its content it is sufficient to write
\begin{minted}{shell}
    op __import__('subprocess').getoutput('cat flag.txt')
\end{minted}
\begin{figure}[H]
    \centering
    \includegraphics[width=0.7\textwidth]{pycalc/pycalc flag.png}
    \caption{Flag}
    \label{fig:PyCalc flag}
\end{figure}


\subsection{Steps to reproduce}
The steps to reproduce are the following:
\begin{itemize}
    \item[1.] Connect to \texttt{10.9.8.1:5006}
    \item[2.] Display the content of the current directory writing
    \begin{minted}{shell}
    op __import__('subprocess').getoutput('ls')
    \end{minted}
    \item[3.] Get the flag printing the content in the \texttt{flag.txt} file
    \begin{minted}{shell}
    op __import__('subprocess').getoutput('cat flag.txt')
    \end{minted}
\end{itemize}

\subsection{Impact}
Using the \texttt{eval()} function to evaluate the arithmetic expressions of a calculator unnecessarily exposes the program to significant risks.
An attacker could use this function's vulnerabilities to view any file in the directory, modify or delete them, move among the directories, etc.
Generally speaking, they can have their own arbitrary code running under the current user privileges.

\subsection{Timing}
It took me some hours to identify the weakness and how to exploit it writing the Python code in a compact way, but it appears that it can be fixed quickly – possibly within 60 days.

% GHIDRA BINARY ANALYSIS
\section{Ghidra binary analysis}
After opening the file with Ghidra and decompiling it, I located the \texttt{main} function.

It is possible to see that the function acquires user input through \texttt{fgets} and stores it in \texttt{local\_28} variable. It then displays it back to the user with a \texttt{printf}.

\begin{figure}[H]
    \centering
    \includegraphics[width=1\textwidth]{ghidra.png}
    \caption{Decompiled main() function with Ghidra}
    \label{fig:Ghidra}
\end{figure}

The vulnerability is in the fact that \texttt{printf} is not a safe function but is exposed to Format String attacks. An attacker could  insert malicious code in the input string, for instance use special characters like \texttt{\%x} to print contents of the stack.

It could be avoided using other functions like \texttt{puts}.

\end{document}