\documentclass{article}
\usepackage[utf8]{inputenc}
\usepackage{titling}

% Page size and margins
\usepackage[a4paper,top=2cm,bottom=2cm,left=3cm,right=3cm,marginparwidth=1.75cm]{geometry}

% Language setting
\usepackage[english]{babel}

% Useful packages
\usepackage{amsmath}
\usepackage[colorlinks=true, allcolors=blue]{hyperref}
\usepackage{minted}

% Header
\usepackage{fancyhdr}
\pagestyle{fancy}
\fancyhead{} % empties all fields
\fancyhead[R]{Francesca Grimaldi}
\fancyhead[L]{IDATT2503 - Exercise 6}

% Images
\usepackage{graphicx}
\graphicspath{ {images/} }
\usepackage{float}
\usepackage{subcaption}

% Tables
\usepackage{tabu}
\usepackage{caption} 
\captionsetup[table]{skip=2pt}
\usepackage[table]{xcolor}
\usepackage{array}

% Lists
\usepackage{listings}
\usepackage{enumitem}
\setlist{topsep=2pt, itemsep=2pt, partopsep=2pt, parsep=2pt}

% renew/new command
\newcommand{\subtitle}[1]{%
  \posttitle{%
    \par\end{center}
    \begin{center}\large#1\end{center}
    \vskip0.5em}
}

% Title and info
\title{%
    \huge Assignment 6}

\subtitle{%
    IDATT2503 - Security in programming and cryptography \\
    Fall 2023
    }

\author{%
  Francesca Grimaldi
}

\date{}


\begin{document}

% TITLE
\maketitle

% INDEX
% \tableofcontents

% TASK 1
\section{Task 1}
For Task 1, I created a \texttt{C++} program that uses \texttt{PBKDF2} with \texttt{SHA1} to crack Ole's password.

As it is possible to see from the code, the \texttt{pbkdf2} function takes as input the given salt and number of iterations, as well as an input password that is built bruteforcing characters from a list of possible ones (in this case, uppercase and lowercase letters of the alphabet).
The result of the hashing is confronted with the given key: if there is a match, it means that the password is the one we were looking for; otherwise, another attempt needs to be made using a different combination of characters.

In this case, I found that the password was \texttt{QwE}.

\begin{figure}[H]
    \centering
    \includegraphics[width=0.85\textwidth]{img/password-cracking.png}
    \caption{Found password}
    \label{fig:password-cracking}
\end{figure}

% TASK 2
\section{Task 2}
For Task 2, I created the web client and server using Vue.
\texttt{PBKDF2} is used for hashing the password and for the authentication.
The homepage presents the Login and Registration components, in which the input fields are validated and the password is hashed (client side) using the JavaScript \texttt{crypto-js} library.
On the server side, Node.js \texttt{Crypto} library is used.

Some screenshots of different scenarios are the following:
\begin{figure}[H]
\centering
\begin{subfigure}[b]{0.45 \textwidth}
    \includegraphics[width=\textwidth]{img/registration-successful.png}
    \caption{Successful registration}
    \label{fig:registration-successful}
\end{subfigure}
\hfill
\begin{subfigure}{0.45 \textwidth}
    \includegraphics[width=\textwidth]{img/login-successful.png}
    \caption{Successful login}
    \label{fig:login-successful}
\end{subfigure}
\caption{Successful login and registration}
\label{fig:success}
\end{figure}

\begin{figure}[H]
\centering
\begin{subfigure}[b]{0.45 \textwidth}
    \includegraphics[width=\textwidth]{img/wrong-username.png}
    \caption{Attempt to login with wrong username}
    \label{fig:wrong-username}
\end{subfigure}
\hfill
\begin{subfigure}{0.45 \textwidth}
    \includegraphics[width=\textwidth]{img/wrong-password.png}
    \caption{Attempt to login with wrong password}
    \label{fig:wrong-password}
\end{subfigure}
\caption{Wrong credentials}
\label{fig:wrong-credentials}
\end{figure}

\begin{figure}[H]
    \centering
    \includegraphics[width=0.45\textwidth]{img/empty-fields.png}
    \caption{Attempt to login with empty fields and to register with already existing username}
    \label{fig:empty-fields}
\end{figure}




\end{document}