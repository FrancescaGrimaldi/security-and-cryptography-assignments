\documentclass{article}
\usepackage[utf8]{inputenc}
\usepackage{titling}

% Page size and margins
\usepackage[a4paper,top=2cm,bottom=2cm,left=3cm,right=3cm,marginparwidth=1.75cm]{geometry}

% Language setting
\usepackage[english]{babel}

% Useful packages
\usepackage{amsmath}
\usepackage[colorlinks=true, allcolors=blue]{hyperref}
\usepackage{minted}

% Header
\usepackage{fancyhdr}
\pagestyle{fancy}
\fancyhead{} % empties all fields
\fancyhead[R]{Francesca Grimaldi}
\fancyhead[L]{IDATT2503 - Exercise 8}

% Images
\usepackage{graphicx}
\graphicspath{ {images/} }
\usepackage{float}

% Tables
\usepackage{tabu}
\usepackage{caption} 
\captionsetup[table]{skip=2pt}
\usepackage[table]{xcolor}
\usepackage{array}

% Lists
\usepackage{listings}
\usepackage{enumitem}
\setlist{topsep=2pt, itemsep=2pt, partopsep=2pt, parsep=2pt}

% renew/new command
\newcommand{\subtitle}[1]{%
  \posttitle{%
    \par\end{center}
    \begin{center}\large#1\end{center}
    \vskip0.5em}
}

% Title and info
\title{%
    \huge Assignment 8}

\subtitle{%
    IDATT2503 - Security in programming and cryptography \\
    Fall 2023
    }

\author{%
  Francesca Grimaldi
}

\date{}


\begin{document}

% TITLE
\maketitle

% INDEX
% \tableofcontents

% TASK 1
\section{Task 1 - AES a little history}

\subsection{The evolutionary history of AES differs from that of DES. Briefly describe the differences of the AES history in comparison to DES.}
The \textbf{Data Encryption Standard (DES)} is a symmetric-key block cipher that uses a 64 bits key: it encrypts data 64 bits at a time.
It was the result of a research project set up by IBM in the late 60s which resulted in a cipher known as LUCIFER.
When it was decided to commercialize it (in the early 70s), IBM introduced significant changes to the algorithm after receiving technical advice from the \textbf{National Security Agency (NSA)}.
This new version was published as an official Federal Information Processing Standard (FIPS) for the United States in 1977, but it was severely criticized for two main reasons:
\begin{itemize}
    \item[1.] The algorithm takes a 64 bit key input, but 8 bits are used for parity checking and are immediately discarded. The actual key is 56-bits long and is considered too short.
    \item[2.] It was thought that S-boxes had some secret trapdoor that could enable the NSA to decrypt messages outside the requirement for the key.
\end{itemize}
Because DES was starting to become vulnerable to brute-force attacks, a new algorithm was required.\\
The \textbf{National Institute of Standards and Technology (NIST)} initiated a public, peer-reviewed competition to choose a new encryption standard. After a process that lasted from 1997 to 2000, the \textbf{Rijndael algorithm} was selected to become the \textbf{Advanced Encryption Standard (AES)}.
Compared to its predecessor, this procedure was noticeably more transparent and open because NIST also asked for input from interested parties on how to choose the algorithm.\\
AES is a symmetric-key block cipher as well, but it supports key lengths of 128, 192, and 256 bits, offering more flexibility and security against modern cryptographic attacks.

\subsection{What is the name of the algorithm that is known as AES?}
The original name of the algorithm that is now known as AES is \textit{Rijndael}.

\subsection{Who developed this algorithm?}
This algorithm was developed by \textit{Vincent Rijmen} and \textit{Joan Daemen}, two Belgian cryptographers.

\subsection{Which block sizes and key lengths are supported by this algorithm?}
Before being chosen as AES, the Rijndael algorithm supported all block lengths and key sizes multiples of 32 bits, between 128 and 256 bits.
After the standardization process, the parameters were restricted to block size of 128 bits and key lengths of 128, 192 or 256 bits.

% TASK 2
\section{Task 2}
The \texttt{task2.py} Python file contains the code used for One-time pad and Affine ciphers, whereas the website \href{https://www.cryptool.org/en/cto/aes-step-by-step}{cryptool.org} was used for AES.

\subsection{Diffusion}
The results of the encryption of both \texttt{x1} and \texttt{x2} using the key \texttt{K1} for different ciphers are the following:

\begin{itemize}
    \item [A)] \texttt{One-time pad (XOR)} \\
    \texttt{x1 -> 0133456789abcdef0123456789abcdef} \\
    \texttt{x2 -> 0103456789abcdef0123456789abcdef} \\
    Two bits change between the plaintexts, and the ciphertexts also differ by two bits. The \textit{diffusion} is very low: there is much statistical correlation between plaintext and ciphertext.
    
    \item [B)] \texttt{Affine cipher} \\
    \texttt{x1 -> e013456789abcdef0123456789abcdef} \\
    \texttt{x2 -> bf03456789abcdef0123456789abcdef} \\
    \textit{Diffusion} increases to a percentage of \texttt{5.46875\%} with the use of affine cipher, as 7 bits change. Although it is a better value than the previous one, it is still low.
    
    \item [C)] \texttt{One round of AES} \\
    \texttt{x1 -> b3543dccec87235705c3aa65640fabdf} \\
    \texttt{x2 -> 834c25e4ec87235705c3aa65640fabdf} \\
    Here we have a diffusion of \texttt{6.25\%} as 8 bits out of 128 change with the changing of 2 bits between the plaintexts. The value is low.
    
    \item [D)] \texttt{Full AES} \\
    \texttt{x1 -> 0694267ba398480c6b2b9f649be476cb} \\
    \texttt{x2 -> 282f7b11019800f8a978c6f750827ab5} \\
    The diffusion in this case is of \texttt{47.65625\%}. Approximately half of the bits of the ciphertext change, which means that the \textit{diffusion} is good.
    
\end{itemize}


\subsection{Confusion}
The results of the encryption of \texttt{x1} using both keys \texttt{K1} and {K2} for different ciphers are the following:

\begin{itemize}
    \item [A)] \texttt{One-time pad (XOR)} \\
    \texttt{x1 using K1 -> 0133456789abcdef0123456789abcdef} \\
    \texttt{x1 using K2 -> 1133456789abcdef0123456789abcdef} \\
    Changing one bit in the key, only one bit changes between the ciphertexts, which is a very low value for the \textit{confusion} property.

    \item [B)] \texttt{Affine cipher} \\
    \texttt{x1 using K1 -> e013456789abcdef0123456789abcdef} \\
    \texttt{x1 using K2 -> f013456789abcdef0123456789abcdef} \\
    Also in this case, \textit{confusion} is very low because changing one bit in the key, only one bit changes between the two ciphertexts.
    
    \item [C)] \texttt{One round of AES} \\
    \texttt{x1 using K1 -> b3543dccec87235705c3aa65640fabdf} \\
    \texttt{x1 using K2 -> eafd942cfc87235715c3aa65740fabdf} \\
    In this case, 18 bits change (in percentage \texttt{14.0625\%}). \textit{Confusion} is better compared to the previous ones but is still a moderate value.
    
    \item [D)] \texttt{Full AES} \\
    \texttt{x1 using K1 -> 0694267ba398480c6b2b9f649be476cb} \\
    \texttt{x1 using K2 -> 7af49a8defad94fa27cb03ac9f1c149a} \\
    The number of different bits between the two ciphertexts is 61, so nearly a half. AES is a cipher with a good level of \textit{confusion}, in the sense that a change in the key affects the calculation of most of the ciphertext.
    
\end{itemize}

% TASK 3
\section{Task 3}
The following subsections contain the outputs of the encryption and decryption of the given messages, obtained using the Caesar cipher at byte level (\texttt{E(x) = x + 3 mod 32}) and CBC mode using bitwise XOR, with initialization vector \texttt{IV = 13}. \\
The code used for this task is available in the \texttt{caesar-cbc.py} Python file.

\subsection{Encryption}
\begin{itemize}
    \item Encryption of \texttt{aaaaaa}: \texttt{oqsuwy}

    \item Encryption of \texttt{dette er en test}: \texttt{llærzåælomfi h,m}
\end{itemize}

\begin{figure}[H]
    \centering
    \includegraphics[width=0.85\textwidth]{img/encryption.png}
    \caption{Output of the encryption}
    \label{fig:encryption}
\end{figure}


\subsection{Decryption}
\begin{itemize}
    \item Decryption of \texttt{qvxæyy hkgdgk,,oqhdnc}: \texttt{cbc mode of operation}
\end{itemize}

\begin{figure}[H]
    \centering
    \includegraphics[width=0.85\textwidth]{img/decryption.png}
    \caption{Output of the decryption}
    \label{fig:decryption}
\end{figure}


\end{document}